\documentclass[10pt,a4paper]{article}
%\usepackage[utf8]{inputenc}
%\usepackage{amsmath}
%\usepackage{amsfonts}
%\usepackage{amssymb}


\begin{document}

%Titelseite
\begin{titlepage}

\begin{center}
\vspace*{1.3cm}
{\Huge Dependable Systems\\(VU 182.712)\\}
\vspace{1.7cm}
{\LARGE Praktisches Ubungsbeispiel SS2015:\\``Zuverl�assigkeitsmodellierung mit \textit{sharpe}''\\}
\vspace{1.7cm}


{\hspace{1cm} Datum der Labor�bung: 18.06.2015}


\begin{table}[h!]
\centering
\begin{tabular}{|p{3.5cm}|p{6.5cm}|}
\hline \textbf{Matr. Nr.} & \textbf{Name} \\
\hline
1228774 &  Schieber Constantin \\
\hline
&  Pacheiner Peter \\
\hline
1226314 & Hofer David \\
\hline
\end{tabular}
\end{table}

\end{center}
\vspace{1.0cm}

\end{titlepage}
\setcounter{page}{2}






\newpage
\section{Abstract}
Ein Computernetzwerk soll durch Markovketten modelliert werden um es auf Fehlersicherheit zu überprüfen.
\\ 
Konkret soll die Mean Time to Failure (MTTF) und die Verfügbarkeit zwischen 2 Systemen evaluiert und verglichen werden. Eines ohne Redundanz und eines mit.
Zusätzlich soll ein Vergleich zwischen den Kosten fuer die beiden Systeme durchgeführt werden.
\section{Aufgabenstellung}
Für beide Systeme gelten die folgenden Fehlerraten ($\lambda$) und Reparaturraten ($\mu$).
\begin{itemize}
	\item $\lambda_R=10^{-4}/Std.$ fuer Rechner
	\item $\lambda_N=2*10^{-5}/Std.$ fuer Switches
	\item  $\mu=10^{-2}/Std. $
	
Um funktionsfaehig zu sein benoetigt das Netzwerk mindestens einen Switch und drei Rechner.
\end{itemize}
\section{Mean Time to Failure}
\subsection{Einfaches System}
Das Computernetzwerk besteht aus drei über einen Switch verbunden Rechnern.
\\
Um die MTTF zu evaluieren reicht es das System durch 2 Zustaende zu modellieren. \\
Einer stellt den funktionierenden Zustand dar, der andere den Fehlerzustand.
Die Wahrscheinlichkeit fuer einen Uebergang in den Fehlerzustand setzt sich dann aus $3*\lambda_R + \lambda_N$ zusammen.
Die Reparaturrate muss nicht berücksichtigt werden da wir uns im Moment ausschließlich fuer die MTTF interessieren.
\subsection{Redundantes System}
\subsection{Vergleich der beiden Systeme}
\section{Availability}
\subsection{Einfaches System}
\subsection{Redundantes System}
\subsection{Vergleich der beiden Systeme}
\section{Kosten}
\end{document}