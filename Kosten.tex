% This file was converted to LaTeX by Writer2LaTeX ver. 1.0.2
% see http://writer2latex.sourceforge.net for more info
\documentclass[a4paper]{article}
\usepackage[ascii]{inputenc}
\usepackage[T1]{fontenc}
\usepackage[english,ngerman]{babel}
\usepackage{amsmath}
\usepackage{amssymb,amsfonts,textcomp}
\usepackage{color}
\usepackage{array}
\usepackage{hhline}
\usepackage{hyperref}
\hypersetup{pdftex, colorlinks=true, linkcolor=blue, citecolor=blue, filecolor=blue, urlcolor=blue, pdftitle=, pdfauthor=, pdfsubject=, pdfkeywords=}
% Page layout (geometry)
\setlength\voffset{-1in}
\setlength\hoffset{-1in}
\setlength\topmargin{2cm}
\setlength\oddsidemargin{2cm}
\setlength\textheight{25.7cm}
\setlength\textwidth{17.001cm}
\setlength\footskip{0.0cm}
\setlength\headheight{0cm}
\setlength\headsep{0cm}
% Footnote rule
\setlength{\skip\footins}{0.119cm}
\renewcommand\footnoterule{\vspace*{-0.018cm}\setlength\leftskip{0pt}\setlength\rightskip{0pt plus 1fil}\noindent\textcolor{black}{\rule{0.25\columnwidth}{0.018cm}}\vspace*{0.101cm}}
% Pages styles
\makeatletter
\newcommand\ps@Standard{
  \renewcommand\@oddhead{}
  \renewcommand\@evenhead{}
  \renewcommand\@oddfoot{}
  \renewcommand\@evenfoot{}
  \renewcommand\thepage{\arabic{page}}
}
\makeatother
\pagestyle{Standard}
\title{}
\author{}
\date{2015-06-18T21:48:05.083000000}
\begin{document}
Abschlie{\ss}end soll noch berechnet werden, ab welchem Verh\"altnis von
Ausfallkosten zu Normalbetriebskosten des Systems, die fehlertolerant
erweiterte Systemvariante zu bevorzugen ist. Laut Angabe ist der Normalbetrieb der erweiterten Variante 2.5 mal
so teuer wie der Normalbetrieb der einfachen Variante. W\"ahrend sich ein System
im ausgefallenen Zustand befindet, verursacht es laufende
Ausfallkosten, die sich auf ein Vielfaches der Betriebskosten der einfachen Variante belaufen.


\bigskip

In den weiteren Berechnung werden die folgenden Bezeichnungen verwendet:


\bigskip

 $K_{\mathit{Normalbetrieb}}$: Kosten f\"ur den Normbalbetrieb des
einfachen Systems

 $K_{\mathit{Ausfall}}$ : laufende Ausfallkosten, im Fall dass sich das
System im Zustand \ {\quotedblbase}Ausgefallen{\textquotedblleft}
befindet.

 $K_{\mathit{GS}}$ : Gesamtkosten f\"ur die simple Systemvariante

 $K_{\mathit{GR}}$ : Gesamtkosten f\"ur die redundante Systemvariante

 $C$ : Vielfachheit der Kosten bei Systemausfall im Vergleich zum Normalbetrieb der einfachen Variante.  
 
 $C=\frac{K_{\mathit{Ausfall}}}{K_{\mathit{Normalbetrieb}}}$ 


\bigskip

 $t_{\mathit{Normalbetrieb}}$ : Zeit in der sich das System laut
Spezifikation verh\"alt

 $t_{\mathit{Ausfall}}$ : Zeit in der sich das System im Zustand
{\quotedblbase}Ausgefallen{\textquotedblleft} befindet.


\bigskip

 $A_{S}$ : \ Verf\"ugbarkeit des einfachen Systems

 $A_{R}$ : \ Verf\"ugbarkeit des redundant aufgebauten Systems.


\bigskip

Die Gesamtkosten f\"ur ein System setzen sich aus den Betriebskosten
w\"ahrend des Normalbetriebs und den Ausfallkosten w\"ahrend der Zeiten
in denen sich das System im ausgefallenen Zustand befindet zusammen.


\bigskip

Damit ergeben sich die folgenden Gesamtkosten:


\bigskip

\begin{equation*}
K_{\mathit{GS}}=K_{\mathit{Normalbetrieb}}\ast
t_{\mathit{Normalbetrieb}}+K_{\mathit{Ausfall}}\ast
t_{\mathit{Ausfall}}
\end{equation*}
\begin{equation*}
K_{\mathit{GR}}=2.5\ast K_{\mathit{Normalbetrieb}}\ast
t_{\mathit{Normalbetrieb}}+K_{\mathit{Ausfall}}\ast
t_{\mathit{Ausfall}}
\end{equation*}

\bigskip

Die Ausfallkosten f\"ur das Systems bleiben gleich, unabh\"angig davon,
welche Variante gew\"ahlt wird und betragen ein Vielfaches der Kosten
des einfachen Systems im Normalbetrieb.

Wir k\"onnen also weiter schreiben:


\bigskip

\begin{equation*}
K_{\mathit{GS}}=K_{\mathit{Normalbetrieb}}\ast
t_{\mathit{Normalbetrieb}}+C\ast K_{\mathit{Normalbetrieb}}\ast
t_{\mathit{Ausfall}}
\end{equation*}

\bigskip

\begin{equation*}
K_{\mathit{GR}}=2.5\ast K_{\mathit{Normalbetrieb}}\ast
t_{\mathit{Normalbetrieb}}+C\ast K_{\mathit{Normalbetrieb}}\ast
t_{\mathit{Ausfall}}
\end{equation*}

\bigskip

Die durchschnittlichen Zeiten f\"ur Normalbetrieb und Ausfall berechnen sich wie folgt:


\bigskip

\begin{equation*}
\mathit{Verf\text{\"u}gbarkeit}=\frac{\mathit{Zeitdauer}\mathit{der}\mathit{Einsatzf\text{\"a}higkeit}}{\mathit{Zeitdauer}\mathit{der}\mathit{Einsatzf\text{\"a}higkeit}+\mathit{Zeitdauer}\mathit{der}\mathit{Nicht}-\mathit{Einsatzf\text{\"a}higkeit}}
\end{equation*}
\begin{equation*}
A=\frac{t_{\mathit{Normalbetrieb}}}{t_{\mathit{Normalbetrieb}}+t_{\mathit{Ausfall}}}
\end{equation*}

$A(t_{\mathit{Normalbetrieb}}+t_{\mathit{Ausfall}})=t_{\mathit{Normalbetrieb}}$


\begin{equation*}
t_{\mathit{Ausfall}}=t_{\mathit{Normalbetrieb}}\ast (\frac{1}{A}-1)
\end{equation*}

\bigskip

Damit ergibt sich:


\bigskip

\begin{equation*}
K_{\mathit{GS}}=K_{\mathit{Normalbetrieb}}\ast
t_{\mathit{Normalbetrieb}}+C\ast K_{\mathit{Normalbetrieb}}\ast
t_{\mathit{Normalbetrieb}}\ast (\frac{1}{A_{S}}-1)
\end{equation*}

\bigskip

\begin{equation*}
K_{\mathit{GS}}=K_{\mathit{Normalbetrieb}}\ast
t_{\mathit{Normalbetrieb}}\ast (1+C\ast (\frac{1}{A_{S}}-1))
\end{equation*}
und


\bigskip

\begin{equation*}
K_{\mathit{GR}}=K_{\mathit{Normalbetrieb}}\ast
t_{\mathit{Normalbetrieb}}\ast (2.5+C\ast (\frac{1}{A_{R}}-1))
\end{equation*}
Die fehlertolerant erweiterte Systemvariante ist zu bevorzugen, sobald gilt:


\bigskip

\begin{equation*}
K_{\mathit{GR}}<K_{\mathit{GS}}
\end{equation*}
Die Konstante berechnet sich daher zu:


\bigskip

\begin{equation*}
(2.5+C\ast (\frac{1}{A_{R}}-1))<(1+C\ast (\frac{1}{A_{S}}-1))
\end{equation*}

\bigskip

\begin{equation*}
1.5+C\ast (\frac{1}{A_{R}}-1)<C\ast (\frac{1}{A_{S}}-1)
\end{equation*}

\bigskip

\begin{equation*}
1.5<C\ast ((\frac{1}{A_{S}}-1)-(\frac{1}{A_{R}}-1))
\end{equation*}

\bigskip

\begin{equation*}
C>\frac{1.5}{(\frac{1}{A_{S}}-1)-(\frac{1}{A_{R}}-1)}
\end{equation*}

\bigskip

\begin{equation*}
C>\frac{1.5}{(\frac{1}{A_{S}}-\frac{1}{A_{R}})}
\end{equation*}
Setzen wir nur die vorher mit Hilfe von SHARPE berechneten Werte f\"ur die Aviability ein erhalten wir:


\bigskip


\bigskip

\begin{equation*}
C>\frac{1.5}{(\frac{1}{0.96899}-\frac{1}{0.99872})}
\end{equation*}

\bigskip

\begin{equation*}
C>48.82693
\end{equation*}
\begin{equation*}
\frac{K_{\mathit{Ausfall}}}{K_{\mathit{Normalbetrieb}}}>48.82693
\end{equation*}

\bigskip

Sobald also die Ausfallkosten des Systems mehr als das 48.83-fache der
Normalbetriebskosten betragen, ist die fehlertolerant erweiterte
Systemvariante zu bevorzugen.


\bigskip
\end{document}
